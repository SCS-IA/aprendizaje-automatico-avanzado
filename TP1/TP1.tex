\documentclass{beamer}

\usepackage[utf8]{inputenc}
\usepackage[T1]{fontenc}
\usepackage[spanish]{babel}
\usepackage{graphicx}
\usepackage{ragged2e}
\usepackage[table]{xcolor}
\usepackage{listings}

\usefonttheme[onlymath]{serif}

\definecolor{codegray}{rgb}{0.5,0.5,0.5}
\definecolor{backcolor}{rgb}{0.95,0.95,0.95}
\definecolor{verdecelda}{HTML}{B7CBA6}
\definecolor{rojocelda}{HTML}{FF9999}
\definecolor{lightgray}{gray}{0.6}


\usetheme{CambridgeUS}
\setbeamertemplate{navigation symbols}{}
\setbeamertemplate{blocks}[rounded][shadow=true]

\lstdefinestyle{mystyle}{
	backgroundcolor=\color{backcolor},
	commentstyle=\color{codegray},
	keywordstyle=\color{blue},
	numberstyle=\tiny\color{codegray},
	stringstyle=\color{red},
	basicstyle=\ttfamily\footnotesize,
	breaklines=true,
	captionpos=b,
	keepspaces=true,
	numbersep=5pt,
	showspaces=false,
	showstringspaces=false,
	showtabs=false,
	tabsize=2,
	inputencoding=utf8,
	extendedchars=true,
}

\lstset{style=mystyle}

\title{Trabajo Práctico 1}
\author{Cisnero, Seivane, Serafini}
\date{22 de Septiembre de 2025}

\begin{document}


\begin{frame}
	\centering
	\includegraphics[width=0.25\textwidth]{UNAHUR (2)}
	\vfill
	{\huge \textbf{Trabajo Práctico 1}}\\[0.2cm]
	{\Large Aprendizaje Automático Avanzado}\\
	\vfill
	{\large Cisnero Matias, Seivane Nicolás, Serafini Franco}\\
	{\small 22 de Septiembre de 2025}
\end{frame}


\section{Ejercicio 1: Creación de Corpus}

\begin{frame}{}
	\centering
	\Large Ejercicio 1: Creación de Corpus\\
	\vspace{0.8cm}
	\includegraphics[width=0.5\textwidth]{imagen_cortazar}
\end{frame}
	
\begin{frame}[fragile]{1.1 Descripción de Librerías Usadas}
	
	\justifying
	Se utilizaron las librerías de $r$ y $pdfplumber$, en la cual utilizamos la ultima para leer página por página de un pdf y la primera para seleccionar las palabras.\\
	\vspace{0.1cm}
	Los links a las librerías son los siguientes.\\
	\vspace{0.1cm}
	\href{https://pypi.org/project/pdfplumber/#extracting-text}{\textbf{pdfplumber}}\\
	\vspace{0.1cm}
	\href{https://docs.python.org/es/3.13/library/re.html}{\textbf{r}}
	\vspace{0.2cm}
	
\begin{block}{Funciones Utilizadas}
	\begin{columns}[c]
		\column{0.45\textwidth}
		\begin{itemize}
			\item \texttt{pdfplumber.open() as pdf}
			\item \texttt{pdf.pages[]}
			\item \texttt{.extract\_text()}
			\item \texttt{.split('\textbackslash n')}
		\end{itemize}
		
		\column{0.45\textwidth}
		\begin{itemize}
			\item \texttt{re.findall()}
			\item \texttt{.endswith()}
			\item \texttt{.strip()}
			\item \texttt{.isdigit()}
			\item \texttt{.split('\textbackslash n')}
		\end{itemize}
	\end{columns}
\end{block}
	
\end{frame}

		%pdfplumber.open() = se utiliza para ir a la dirección del pdf, retornando una instancia de la% clase $pdfplumber.PDF$ 
%$pdf.pages[]$ = es una propiedad de la clase $pdfplumber.PDF$, la cual se puede indexar para acceder a %las paginas del pdf representadas en la clase $pdfplumber.Page$
%$.extract_text()$ = Método de la clase $pdfplumber.Page$, recopila todos los objetos de caracteres de %la página en un sol string.
%$split('\_n')$ = Divide el string que se genero antes en los saltos de pagina, generando una lista de %lineas.
%$re.findall()$ =
%$.endswith()$ =
%$.strip()$ =
%$.isdigit()$ =

\begin{frame}[fragile]{1.2.1 Estructura de Código}
	
	\justifying
	\textbf{\underline{Se utiliza la siguiente estructura de codigo:}}\\
	\vspace{0.1cm}
	Se comienza importando las librerías y creando una lista de palabras, donde se irán agregando las extracciones de texto.
	\begin{lstlisting}[language=Python]
import pdfplumber
import re

words = []
	\end{lstlisting}
	\justifying
	En lo cual se sigue utilizando la función \texttt{pdfplumber.open() as pdf}, en la cual se debe especificar la ruta hacia el pdf. El cual nos devuelve $pdf$ como una instancia de la clase \texttt{pdfplumber.PDF}
		\begin{lstlisting}[language=Python]
with pdfplumber.open("ruta") as pdf:
	\end{lstlisting}
\end{frame}

	
\begin{frame}[fragile]{1.2.2 Estructura de Código}
	
	\justifying
	Se continua utilizando una propiedad de la clase \texttt{pdfplumber.Page}, de la cual se puede indexar para acceder a las paginas del pdf
	\begin{lstlisting}[language=Python]
with pdfplumber.open("ruta") as pdf:
	for page in pdf.pages[:]:
	\end{lstlisting}
	\justifying
	En lo cual se utiliza el metodo \texttt{.extract\_text()}, que recopila todos los objetos de caracteres de la página en un solo string.
	\begin{lstlisting}[language=Python]
with pdfplumber.open("ruta") as pdf:
	for page in pdf.pages[:]:
		text = page.extract_text()
		if text:
	\end{lstlisting}
\end{frame}
	
\begin{frame}[fragile]{1.2.3 Estructura de Código}
	
	\justifying
	Se continua diviendo el string segun el metodo \texttt{.split('\textbackslash n')}, el cual devuleve una lista de strings, los cuales fueron separados de acuerdo a \textbackslash n, ergo saltos de linea.
	\begin{lstlisting}[language=Python]
with pdfplumber.open("ruta") as pdf:
	for page in pdf.pages[:]:
		text = page.extract_text()
			if text:
				lines = text.split('\n')
	\end{lstlisting}
	\justifying
	Luego se sacan las lineas que sean numeros de pagina tanto en el pie de la misma como en el encabezado. La forma de extraccion varia de acuerdo a como es el pdf.
	\begin{lstlisting}[language=Python]
				if lines[-1].strip().isdigit():
					lines = lines[:-1]
				if lines[0].strip().isdigit():
					lines = lines[1:]
				
	\end{lstlisting}
\end{frame}
	
\begin{frame}[fragile]{1.2.4 Estructura de Código}
	
	\justifying
	Se crea por linea una lista con el método de la librería r:\\
	\vspace{0.1cm}
	\texttt{re.findall(r"\_w+|[.,!?;:]", line)}, en el cual se separan con expresiones regulares las palabras con \textbackslash w+ y aparte los signos de puntuación con [.,!?;:], en una lista de strings. Luego para cada palabra se la pasa a minúscula con el método \texttt{.lower()}.
	\begin{lstlisting}[language=Python]
with pdfplumber.open("ruta") as pdf:
	for page in pdf.pages[:]:
		text = page.extract_text()
		if text:
			lines = text.split('\n')
			if lines[-1].strip().isdigit():
				lines = lines[:-1]
			if lines[0].strip().isdigit():
				lines = lines[1:]
			for line in lines:
				tokens = re.findall(r"\w+|[.,!?;:]", line)
				tokens = [token.lower() for token in tokens]
	\end{lstlisting}

\end{frame}

	
\begin{frame}[fragile]{1.2.5 Estructura de Código}
	
	\justifying
	Luego se diferencia por linea los puntos aparte, los cuales consideeramos los ultimos puntos de las lineas. Cada linea, las cuales fueron convertidas en listas de strings son agregadas a la lista del corpus\\

	\begin{lstlisting}[language=Python]
with pdfplumber.open("ruta") as pdf:
	for page in pdf.pages[:]:
		text = page.extract_text()
			if text:
				lines = text.split('\n')
				if lines[-1].strip().isdigit():
					lines = lines[:-1]
				if lines[0].strip().isdigit():
					lines = lines[1:]
				for line in lines:
					tokens = re.findall(r"\_w+|[.,!?;:]", line)
					tokens = [token.lower() for token in tokens]
				if line.endswith("."):
					tokens[-1]= ". "
				words.extend(tokens)
	\end{lstlisting}
	
	
%%% RAYUELA LOCOOOO	
	
	
	
\end{frame}
	
\begin{frame}{1.3 Libros utilizados: Rayuela}
	\justifying
	\textbf{Titulo:} Rayuela\\
	\textbf{Autor:} Julio Cortazar\\
	\textbf{Año :} 1963\\
	Se extrayeron 197.342 caracteres y 20.810 caracteres únicos que conforman el vocabulario.\\
	\centering
	\vspace{0.2cm}
	\includegraphics[width=0.3\textwidth]{rayuela_cortazar}
	
\end{frame}
	
\begin{frame}[fragile]{1.3.2 Código utilizado: Rayuela}
\begin{lstlisting}[language=Python]
with pdfplumber.open("Julio-Cortazar-Rayuela.pdf") as pdf:
	for page in pdf.pages[7:]:
		text = page.extract_text()
			if text:
				lines = text.split('\n')
				if lines[-1].strip().isdigit():
					lines = lines[:-1]
				if lines[0].strip().isdigit():
					lines = lines[1:]
				if lines[0].strip().isdigit():
					lines = lines[1:]
				if lines[-1].strip().isdigit():
					lines = lines[:-1]
				for line in lines:
					tokens = re.findall(r"\w+|[.,!?;:]", line)
					tokens = [token.lower() for token in tokens]
				if line.endswith("."):
					tokens[-1]= ". "
				words.extend(tokens)
\end{lstlisting}
	
\end{frame}
	
	
\begin{frame}{1.3.3 Ejemplo Borrado: Rayuela}
	\centering
	\includegraphics[width=0.5\textwidth]{borrado_rayuela}
	
\end{frame}	
	
	
%%% TODOS LOS FUEGOS

\begin{frame}{1.3 Libros utilizados: Todos los fuegos}
\justifying
\textbf{Titulo:} Todos los fuegos el fuego.\\
\textbf{Autor:} Julio Cortazar\\
\textbf{Año :} 1966\\
Se extrayeron 55.948 caracteres y 2.828 caracteres únicos que conforman el vocabulario.\\
\centering
\vspace{0.2cm}
\includegraphics[width=0.3\textwidth]{todos_los_fuegos_cortazar}

\end{frame}

\begin{frame}[fragile]{1.3.2 Código utilizado: Todos los fuegos}
\begin{lstlisting}[language=Python]
with pdfplumber.open("Julio Cortazar Todos los fuegos.pdf") as pdf:
	for page in pdf.pages[:-1]:
		text = page.extract_text()
			if text:
				lines = text.split('\n')
				if lines[-1].strip().isdigit():
					lines = lines[:-1]
				for line in lines:
					tokens = re.findall(r"\w+|[.,!?;:]", line)
					tokens = [token.lower() for token in tokens]
				if line.endswith("."):
					tokens[-1]= ". "
				words.extend(tokens)
		\end{lstlisting}

\end{frame}


\begin{frame}{1.3.3 Ejemplo Borrado: Todos los fuegos}
\centering
\includegraphics[width=0.5\textwidth]{borrado_todoslosfuegos}

\end{frame}	
	
%%% un tal lucas

\begin{frame}{1.3 Libros utilizados: Historias de cronopios y de famas}
	\justifying
	\textbf{Titulo:} Historias de cronopios y de famaso.\\
	\textbf{Autor:} Julio Cortazar\\
	\textbf{Año :} 1962\\
	Se extrayeron 32.224 caracteres y 2.514 caracteres únicos que conforman el vocabulario.\\
	\centering
	\vspace{0.2cm}
	\includegraphics[width=0.3\textwidth]{historias_de_cronopios_y_de_famas_cortazar}
	
\end{frame}

\begin{frame}[fragile]{1.3.2 Código utilizado: Historias de cronopios y de famas}
	\justifying
	En este caso no fue necesario quitar ninguna linea.
	\begin{lstlisting}[language=Python]
with pdfplumber.open("Historias-de-Cronopios-y-de-Famas - Julio Cortazar.pdf") as pdf:
	for page in pdf.pages[3:-1]:
		text = page.extract_text()
			if text:
				lines = text.split('\n')
				for line in lines:
					tokens = re.findall(r"\w+|[.,!?;:]", line)
					tokens = [token.lower() for token in tokens]
				if line.endswith("."):
					tokens[-1]= ". "
				words.extend(tokens)
	\end{lstlisting}
	
\end{frame}
	
	
%% ahora si un tal lucas

\begin{frame}{1.3 Libros utilizados: Un tal Lucas.}
	\justifying
	\textbf{Titulo:} Un tal Lucas.\\
	\textbf{Autor:} Julio Cortazar\\
	\textbf{Año :} 1979\\
	Se extrayeron 32.224 caracteres y 2.514 caracteres únicos que conforman el vocabulario.\\
	\centering
	\vspace{0.2cm}
	\includegraphics[width=0.3\textwidth]{un_tal_lucas_cortazar}
	
\end{frame}

\begin{frame}[fragile]{1.3.2 Código utilizado: Un tal Lucas.}
	\begin{lstlisting}[language=Python]

with pdfplumber.open("Lucas_Julio_Cortazar.pdf") as pdf:
	for page in pdf.pages[5:]:
		text = page.extract_text()
			if text:
				lines = text.split('\n')
				if lines[-1].strip().isdigit():
					lines = lines[:-1]
				lines = lines[1:]
				for line in lines:
					tokens = re.findall(r"\w+|[.,!?;:]", line)
					tokens = [token.lower() for token in tokens]
				if line.endswith("."):
					tokens[-1]= ". "
				words.extend(tokens)
	\end{lstlisting}
	
\end{frame}


\begin{frame}{1.3.3 Ejemplo Borrado: Un tal Lucas.}
	\centering
	\includegraphics[width=0.5\textwidth]{borrado_un_tal_lucas}
\end{frame}	
	
	
\begin{frame}[fragile]{1.4 Corpus final}
	\justifying
	Se guarda el corpus final en un archivo llamado \textbf{corpus.txt}.
	
	\begin{lstlisting}[language=Python]
		with open("corpus.txt", "w", encoding="utf-8") as f:
			f.write("\n".join(words))
	\end{lstlisting}
	
	Obteniendo un corpus como se ve en la siguiente imagen.
	\vspace{0.1cm}
\begin{columns}[t] % [t] para alinear arriba
	\column{0.55\textwidth}
	\textbf{Vocabulario (único):} 27.971\\
	\textbf{Corpus total:} 310.347
	
	\column{0.45\textwidth}
	\includegraphics[width=0.5\linewidth]{corpus_txt.png}
\end{columns}

\end{frame}


\section{Ejercicio 2: Implementación CBOW y SkipGram}

\begin{frame}{}
	\centering
	\Large Ejercicio 2: Implementación CBOW y SkipGram\\
	\vspace{0.8cm}
	\includegraphics[width=0.5\textwidth]{imagenes_cbow_skipgram}
\end{frame}



	
\begin{frame}[fragile]{2.1 Pasos previos}
	\justifying
	Se abre y carga el \textbf{corpus.txt}.
	
	\begin{lstlisting}[language=Python]
with open("corpus.txt", "r", encoding="utf-8") as f:
	corpus = f.read().splitlines()
	\end{lstlisting}
	Luego se crean los diccionarios que se utilizaran en ambos métodos.
	\begin{lstlisting}[language=Python]
    vocab = sorted(set(corpus))
		vocab_tamano = len(vocab)
		palabra_a_indice = {palabra: i for i, palabra in enumerate(vocab)}
		indice_a_palabra = {i: palabra for i, palabra in enumerate(vocab)}
	\end{lstlisting}

\end{frame}
	
\begin{frame}[fragile]{2.2 CBOW}
	\begin{block}{\textbf{Definición:} Continuous Bag of Words(CBOW)}
	\justifying
	\vspace{0.1cm}
	\textbf{Propósito:} Es un modelo de aprendizaje automático para aprender representaciones de palabras que capturan el "significado" de las palabras basadas en su contexto.\\
	\vspace{0.1cm}
	\textbf{Principio:} A diferencia de los modelos más simples, CBOW utiliza un contexto de $C$ palabras para predecir una palabra central\\
	\vspace{0.1cm}
	\textbf{Contexto vs. Predicción:}  A partir de un contexto de $C$ palabras ($p_{I,1}, p_{I,2}, ..., p_{I,C}$), se intenta predecir la palabra objetivo ($p_O$), que generalmente es la palabra central
\end{block}
	
	
	

	
\end{frame}
	
	
	
	
	
	
	
	
	
	
	
	
\end{document}