\documentclass[12pt,a4paper]{article}
\usepackage[spanish]{babel}
\usepackage[utf8]{inputenc}
\usepackage[T1]{fontenc}
\usepackage{geometry}
\usepackage{setspace}
\usepackage{titling}
\usepackage{parskip}
\usepackage{fancyhdr}
\usepackage{graphicx}
\usepackage{multicol}
\usepackage{booktabs}
\usepackage{array}
\usepackage{courier}
\usepackage{hyperref}
\usepackage{amsmath}
\geometry{margin=2.3cm}
\setstretch{1.0}
\renewcommand{\familydefault}{\rmdefault}

\pretitle{\begin{center}\LARGE\bfseries}
	\posttitle{\par\end{center}\vskip 1em}
\preauthor{\begin{center}\large}
	\postauthor{\end{center}}



\title{Informe de Resultados — Trabajo Práctico 1\\[0.5em]
	\large Materia: Aprendizaje Automático Avanzado}
\author{
	Profesores: Dr.~Juan Santos, Lic.~Elimiano Churruca\\[0.3em]
	Alumnos: Franco Serafini, Seivane Nicolás, Matías Cisnero
}
\date{}

\pagestyle{fancy}
\fancyhf{}
\fancyhead[L]{Aprendizaje Automático Avanzado}
\fancyhead[R]{Trabajo Práctico 1}
\fancyfoot[C]{\thepage}

\begin{document}
	\maketitle
	\thispagestyle{empty}
	
	\section*{Resumen general}
	
	\subsection*{Métricas Utilizadas}
	
	\noindent Para la realización del siguiente informe se utilizará la métrica \textit{Accuracy}, la cual representa el porcentaje de predicciones correctas respecto al total de predicciones realizadas. En este caso, se introduce una ligera modificación en su aplicación. Para la predicción de la palabra central, se considera el valor con mayor probabilidad obtenido por la función \textit{soft-max} al momento de la inferencia, lo que corresponde a la métrica \textbf{Top1}. Asimismo, se evaluará la métrica \textbf{Top5}, que toma en cuenta los cinco valores más altos devueltos por la función \textit{soft-max}, considerando una predicción correcta si la palabra esperada se encuentra dentro de esos cinco primeros resultados.
	
	\begin{equation}
		\text{Accuracy} = \frac{\text{Número de predicciones correctas}}{\text{Número total de muestras}}
		\label{eq:accuracygeneral}
	\end{equation}
	
	\subsection*{Corpus sin Fragmentar}
	
	\noindent\textbf{Características del corpus:} 
	Se realizó un método de obtención de \textit{Representaciones contextuales, CBOW}, donde se tomo a cada palabra y signo de puntuación como entrada del vocabulario, dando un total de 310.347 palabras en el corpus y un vocabulario de 27.971 tokens.
	
	
	
	\noindent\textbf{Configuración del modelo 1:}
	\begin{itemize}
		\item Neuronas ocultas: 130
		\item Tamaño de contexto: 2
		\item Épocas: 1250
		\item Tasa de aprendizaje: 0.001
		\item Palabras mal predichas: 3790
	\end{itemize}
	
	\noindent\textbf{Resultados finales del modelo 1:}
	

	\begin{center}
		\begin{tabular}{@{}ll@{}}
			\toprule
			\textbf{Métrica} & \textbf{Valor} \\ 
			\midrule
			Top1 & 67.72\% \\
			Top5 & 86.32\% \\
			\bottomrule
		\end{tabular}
	\end{center}
	
	
	\noindent\textbf{Configuración del modelo 2:}
	\begin{itemize}
		\item Neuronas ocultas: 100
		\item Tamaño de contexto: 4
		\item Épocas: 1600
		\item Tasa de aprendizaje: 0.001
	\end{itemize}
	
	\noindent\textbf{Resultados finales del modelo 2:}\\
	

	\begin{center}
		\begin{tabular}{@{}ll@{}}
			\toprule
			\textbf{Métrica} & \textbf{Valor} \\ 
			\midrule
			Top1 & 65.79\% \\
			Top5 & 86.01\% \\
			\bottomrule
		\end{tabular}
	\end{center}
	
	\newpage
	\section*{Palabras que no se predijeron bien}
	\begin{multicols}{3}
		\footnotesize
		\begin{verbatim}
.              2818
,              2242
y              1552
.              1316
que            1085
de             1068
no              782
a               776
en              669
pero            576
lo              479
la              478
el              430
con             417
se              384
como            373
es              357
por             350
o               308
?               299
para            299
era             292
un              282
una             281
yo              247
cuando          244
más             240
oliveira        238
eso             235
si              234
le              216
ya              212
dijo            205
todo            203
me              191
porque          181
del             179
los             171
al              165
así             148
;               144
qué             142
su              140
:               129
nada            123
él              123
estaba          123
siempre         122
hasta           122
después         122
algo            121
vos             119
las             118
ahora           117
te              117
talita          112
maga            111
mismo           108
sí              102
horacio          99
poco             99
está             98
ahí              97
había            95
otro             94
bien             91
noche            90
menos            90
traveler         90
ella             86
vez              86
sin              85
mejor            85
dos              84
entonces         84
casi             83
hay              82
uno              81
tiempo           80
también          79
otra             74
donde            72
mano             69
tenía            69
hacer            68
tan              68
ese              67
vida             66
aunque           65
tiene            63
casa             63
sobre            62
ser              62
usted            61
esa              61
aquí             60
decir            59
ronald           57
cosa             57
mi               57
mamá             57
cara             55
parece           55
todavía          54
cosas            54
viejo            54
va               53
esto             53
gregorovius      53
ver              52
día              51
tarde            51
hombre           50
solamente        49
realidad         48
hace             48
mundo            48
tanto            47
mientras         47
sé               47
entre            45
rocamadour       45
mucho            45
dice             44
fuera            44
cama             44
boca             44
etienne          44
fue              43
puerta           43
verdad           42
babs             42
veces            42
desde            42
antes            41
claro            41
pensar           40
lado             40
mal              40
mí               40
ventana          39
aire             39
pieza            39
sueño            39
ni               39
fin              38
nos              38
mañana           38
mirando          38
manú             38

			
		\end{verbatim}
	\end{multicols}
	
	\newpage
	\section*{Análisis de contextos}
	
	\noindent Considerando los resultados obtenidos y tras realizar pruebas con distintos modelos que presentaron valores similares de acierto y error, se decidió analizar las frecuencias de las palabras que fueron incorrectamente predichas. A continuación, se muestran aquellas palabras que se predicen de forma errónea en más de 500 ocasiones, junto con las palabras que con mayor frecuencia aparecen a su derecha e izquierda. El objetivo de este análisis es combinar esta información posteriormente con el fin de mejorar el desempeño general del modelo.
	
\begin{multicols}{2}
	\raggedcolumns
	\noindent\textbf{Palabras más frecuentes cerca de \texttt{'.'} (2818 apariciones):}
	\begin{verbatim}
		.: 1910
		no: 421
		pero: 324
		la: 324
		y: 318
	\end{verbatim}
	
	\noindent\textbf{Palabras más frecuentes cerca de \texttt{','} (2242 apariciones):}
	\begin{verbatim}
		y: 1992
		pero: 755
		la: 696
		que: 609
		el: 576
	\end{verbatim}
	
	\noindent\textbf{Palabras más frecuentes cerca de \texttt{'y'} (1552 apariciones):}
	\begin{verbatim}
		,: 1992
		el: 411
		la: 383
		se: 323
		.: 318
	\end{verbatim}
	
	\noindent\textbf{Palabras más frecuentes cerca de \texttt{'. '} (1316 apariciones):}
	\begin{verbatim}
		no: 227
		.: 204
		oliveira: 116
		la: 113
		por: 112
	\end{verbatim}
	
	\noindent\textbf{Palabras más frecuentes cerca de \texttt{'que'} (1085 apariciones):}
	\begin{verbatim}
		lo: 788
		,: 609
		se: 571
		no: 503
		de: 475
	\end{verbatim}
	
	\noindent\textbf{Palabras más frecuentes cerca de \texttt{'de'} (1068 apariciones):}
	\begin{verbatim}
		la: 1854
		los: 659
		,: 484
		que: 475
		las: 466
	\end{verbatim}
	
	\noindent\textbf{Palabras más frecuentes cerca de \texttt{'no'} (782 apariciones):}
	\begin{verbatim}
		,: 553
		que: 503
		.: 421
		se: 364
		. : 227
	\end{verbatim}
	
	\noindent\textbf{Palabras más frecuentes cerca de \texttt{'a'} (776 apariciones):}
	\begin{verbatim}
		la: 977
		,: 417
		.: 253
		y: 249
		los: 245
	\end{verbatim}
	
	\noindent\textbf{Palabras más frecuentes cerca de \texttt{'en'} (669 apariciones):}
	\begin{verbatim}
		el: 1219
		la: 1101
		,: 435
		que: 394
		un: 336
	\end{verbatim}
	
	\noindent\textbf{Palabras más frecuentes cerca de \texttt{'pero'} (576 apariciones):}
	\begin{verbatim}
		,: 755
		.: 324
		no: 145
		. : 91
		el: 58
	\end{verbatim}
\end{multicols}
	
	
	\subsection*{Corpus Fragmentado por Frecuencias}
	
	\noindent\textbf{Características del corpus:} 
	Se utilizó un modelo de fragmentación explicado anteriormente, dando un total de 289.919 palabras en el corpus y un vocabulario de 28.030 tokens. Se agruparon en un único \textit{token} aquellas palabras que aparecían más de 500 veces y que, además, compartían un contexto inmediato (de una posición de distancia) en más de 200 ocasiones. Este proceso se aplicó de forma iterativa hasta que no fue posible realizar más combinaciones (\textit{merges}).
	

	\noindent\textbf{Configuración del modelo:}
	\begin{itemize}
		\item Neuronas ocultas: 130
		\item Tamaño de contexto: 5
		\item Épocas: 1000
		\item Tasa de aprendizaje: 0.01
		\item Palabras mal predichas: 132
	\end{itemize}
	
	\noindent\textbf{Resultados finales:}
	
	\begin{center}
		\begin{tabular}{@{}ll@{}}
			\toprule
			\textbf{Métrica} & \textbf{Valor} \\ 
			\midrule
			Top1 & 94.50\% \\
			Top5 & 99.10\% \\
			\bottomrule
		\end{tabular}
	\end{center}
	

	
	
	\newpage
	\section*{Palabras que no se predijeron bien}
	\begin{multicols}{3}
		\footnotesize
		\begin{verbatim}
,                450
.                321
.                307
de               127
y                109
que              105
a                101
no                92
la                68
se                55
en                47
el                45
por               41
es                37
me                36
, y               36
lo                36
talita            35
con               32
?                 27
un                26
yo                25
era               17
eso               17
o                 15
del               14
una               14
todo              14
oliveira          13
como              13
para              13
le                11
horacio           11
más               11
te                11
ya                10
dijo              10
traveler          10
. .                9
vos                9
si                 8
, pero             8
al                 8
. . .              7
de la              7
sí                 7
gregorovius        7
lo que             6
él                 6
a la               6
siempre            6
los                5
, no               5
decir              5
mi                 4
bien               4
porque             4
sin                4
después            4
ella               4
otro               4
qué                3
cuando             3
mejor              3
estaba             3
babs               3
menos              3
tiempo             3
, que              3
mismo              3
ahora              2
:                  2
dos                2
tarde              2
vez                2
así                2
su                 2
pero               2
había              2
otra               2
en el              2
las                2
hay                2
algo               2
dijo oliveira      2
va                 2
sé                 2
está               2
ronald             2
hablar             1
entonces           1
hacer              1
, en               1
no ,               1
fuera              1
de que             1
hombre             1
aquí               1
sabe               1
casi               1
que no             1
mucho              1
ossip              1
wong               1
empezó             1
suerte             1
desde              1
etienne            1
. no               1
usted              1
miedo              1
viejo              1
acuerdo            1
cosas              1
entrar             1
apenas             1
maga               1
quiero             1
manú               1
quiere             1
donde              1
, de               1
pensar             1
parece             1
izquierda          1
estar              1
calor              1
todavía            1
poco               1
esto               1
ahí                1
aunque             1

		\end{verbatim}
	\end{multicols}
	
		\newpage
	\section*{Análisis de contextos}
	
	\begin{multicols}{2}
		\raggedcolumns
		
		\noindent\textbf{Palabras más frecuentes cerca de \texttt{,} (450 apariciones):}
		\begin{verbatim}
			una: 279
			los: 265
			por: 241
			es: 232
			como: 222
		\end{verbatim}
		
		\noindent\textbf{Palabras más frecuentes cerca de \texttt{. } (321 apariciones):}
		\begin{verbatim}
			no: 201
			. .: 190
			la: 112
			por: 112
			y: 97
		\end{verbatim}
		
		\noindent\textbf{Palabras más frecuentes cerca de \texttt{.} (307 apariciones):}
		\begin{verbatim}
			a: 224
			en: 192
			por: 169
			es: 143
			se: 141
		\end{verbatim}
		
		\noindent\textbf{Palabras más frecuentes cerca de \texttt{de} (127 apariciones):}
		\begin{verbatim}
			su: 194
			y: 149
			después: 107
			.: 102
			lo: 100
		\end{verbatim}
		
		\noindent\textbf{Palabras más frecuentes cerca de \texttt{y} (109 apariciones):}
		\begin{verbatim}
			la: 285
			se: 272
			los: 188
			a: 159
			de: 149
		\end{verbatim}
		
		\noindent\textbf{Palabras más frecuentes cerca de \texttt{que} (105 apariciones):}
		\begin{verbatim}
			en: 247
			me: 195
			es: 183
			le: 179
			hay: 178
		\end{verbatim}
		
		\noindent\textbf{Palabras más frecuentes cerca de \texttt{a} (101 apariciones):}
		\begin{verbatim}
			los: 235
			.: 224
			va: 209
			un: 195
			las: 191
		\end{verbatim}
		
		\noindent\textbf{Palabras más frecuentes cerca de \texttt{no} (92 apariciones):}
		\begin{verbatim}
			se: 262
			. : 201
			es: 133
			ya: 122
			me: 115
		\end{verbatim}
		
		\noindent\textbf{Palabras más frecuentes cerca de \texttt{la} (68 apariciones):}
		\begin{verbatim}
			y: 285
			maga: 271
			por: 266
			con: 265
			dijo: 146
		\end{verbatim}
		
		\noindent\textbf{Palabras más frecuentes cerca de \texttt{se} (55 apariciones):}
		\begin{verbatim}
			y: 272
			no: 262
			.: 141
			le: 109
			había: 104
		\end{verbatim}
		
		\noindent\textbf{Palabras más frecuentes cerca de \texttt{en} (47 apariciones):}
		\begin{verbatim}
			que: 247
			los: 229
			una: 225
			.: 192
			las: 177
		\end{verbatim}
		
		\noindent\textbf{Palabras más frecuentes cerca de \texttt{el} (45 apariciones):}
		\begin{verbatim}
			por: 232
			con: 223
			todo: 119
			, y: 96
			. : 91
		\end{verbatim}
		
		\noindent\textbf{Palabras más frecuentes cerca de \texttt{por} (41 apariciones):}
		\begin{verbatim}
			la: 266
			qué: 262
			,: 241
			el: 232
			.: 169
		\end{verbatim}
		
		\noindent\textbf{Palabras más frecuentes cerca de \texttt{es} (37 apariciones):}
		\begin{verbatim}
			,: 232
			que: 183
			.: 143
			no: 133
			un: 131
		\end{verbatim}
		
		\noindent\textbf{Palabras más frecuentes cerca de \texttt{me} (36 apariciones):}
		\begin{verbatim}
			que: 195
			,: 167
			.: 126
			no: 115
			y: 80
		\end{verbatim}
		
		\noindent\textbf{Palabras más frecuentes cerca de \texttt{, y} (36 apariciones):}
		\begin{verbatim}
			el: 96
			la: 86
			que: 76
			en: 62
			después: 59
		\end{verbatim}
		
		\noindent\textbf{Palabras más frecuentes cerca de \texttt{lo} (36 apariciones):}
		\begin{verbatim}
			a: 162
			mejor: 157
			,: 153
			por: 118
			mismo: 105
		\end{verbatim}
		
		\noindent\textbf{Palabras más frecuentes cerca de \texttt{talita} (35 apariciones):}
		\begin{verbatim}
			dijo: 92
			.: 89
			,: 73
			. : 48
			a: 37
		\end{verbatim}
		
		\noindent\textbf{Palabras más frecuentes cerca de \texttt{con} (32 apariciones):}
		\begin{verbatim}
			la: 265
			el: 223
			un: 201
			una: 140
			los: 123
		\end{verbatim}
		
		\noindent\textbf{Palabras más frecuentes cerca de \texttt{?} (27 apariciones):}
		\begin{verbatim}
			no: 69
			qué: 40
			dijo: 29
			,: 25
			preguntó: 24
		\end{verbatim}
		
		\noindent\textbf{Palabras más frecuentes cerca de \texttt{un} (26 apariciones):}
		\begin{verbatim}
			poco: 277
			como: 224
			con: 201
			a: 195
			es: 131
		\end{verbatim}
		
		\noindent\textbf{Palabras más frecuentes cerca de \texttt{yo} (25 apariciones):}
		\begin{verbatim}
			,: 171
			.: 141
			que: 90
			y: 78
			no: 72
		\end{verbatim}
		
		\noindent\textbf{Palabras más frecuentes cerca de \texttt{era} (17 apariciones):}
		\begin{verbatim}
			,: 110
			que: 88
			no: 79
			un: 77
			una: 72
		\end{verbatim}
		
		\noindent\textbf{Palabras más frecuentes cerca de \texttt{eso} (17 apariciones):}
		\begin{verbatim}
			,: 114
			que: 96
			todo: 79
			por: 70
			es: 62
		\end{verbatim}
		
		\noindent\textbf{Palabras más frecuentes cerca de \texttt{o} (15 apariciones):}
		\begin{verbatim}
			,: 194
			de: 58
			el: 38
			dos: 37
			en: 35
		\end{verbatim}
		
		\noindent\textbf{Palabras más frecuentes cerca de \texttt{del} (14 apariciones):}
		\begin{verbatim}
			,: 52
			mundo: 47
			lado: 46
			colorado: 39
			tiempo: 35
		\end{verbatim}
		
		\noindent\textbf{Palabras más frecuentes cerca de \texttt{una} (14 apariciones):}
		\begin{verbatim}
			,: 279
			en: 225
			con: 140
			como: 134
			es: 118
		\end{verbatim}
		
		\noindent\textbf{Palabras más frecuentes cerca de \texttt{todo} (14 apariciones):}
		\begin{verbatim}
			el: 119
			,: 91
			eso: 79
			.: 78
			de: 74
		\end{verbatim}
		
		\noindent\textbf{Palabras más frecuentes cerca de \texttt{oliveira} (13 apariciones):}
		\begin{verbatim}
			,: 115
			.: 113
			a: 76
			se: 70
			. : 66
		\end{verbatim}
		
		\noindent\textbf{Palabras más frecuentes cerca de \texttt{como} (13 apariciones):}
		\begin{verbatim}
			un: 224
			,: 222
			si: 211
			una: 134
			es: 84
		\end{verbatim}
		
		\noindent\textbf{Palabras más frecuentes cerca de \texttt{para} (13 apariciones):}
		\begin{verbatim}
			que: 147
			,: 101
			.: 64
			la: 47
			no: 44
		\end{verbatim}
		
		\noindent\textbf{Palabras más frecuentes cerca de \texttt{le} (11 apariciones):}
		\begin{verbatim}
			que: 179
			,: 131
			se: 109
			y: 92
			no: 91
		\end{verbatim}
		
		\noindent\textbf{Palabras más frecuentes cerca de \texttt{horacio} (11 apariciones):}
		\begin{verbatim}
			,: 102
			. : 38
			.: 29
			a: 23
			se: 20
		\end{verbatim}
		
		\noindent\textbf{Palabras más frecuentes cerca de \texttt{más} (11 apariciones):}
		\begin{verbatim}
			que: 161
			,: 71
			vez: 67
			lo: 59
			bien: 59
		\end{verbatim}
		
		\noindent\textbf{Palabras más frecuentes cerca de \texttt{te} (11 apariciones):}
		\begin{verbatim}
			que: 100
			no: 79
			.: 61
			,: 58
			vos: 38
		\end{verbatim}
		
	\end{multicols}
	
	
	\subsection*{Corpus Fragmentado con BPE}
	
	\noindent\textbf{Características del corpus:} 
	Se utilizó un modelo de fragmentación BPE con 10.000 merges, dando un total de 359.142 palabras en el corpus y un vocabulario de 9.676 tokens.
	
	\noindent\textbf{Configuración del modelo:}
	\begin{itemize}
		\item Neuronas ocultas: 130
		\item Tamaño de contexto: 15
		\item Épocas: 2000
		\item Tasa de aprendizaje: 0.01
		\item Palabras mal predichas: 136
	\end{itemize}
	
	\noindent\textbf{Resultados finales:}
	
	\begin{center}
		\begin{tabular}{@{}ll@{}}
			\toprule
			\textbf{Métrica} & \textbf{Valor} \\ 
			\midrule
			Top1 & 81.10\% \\
			Top5 & 95.53\% \\
			\bottomrule
		\end{tabular}
	\end{center}
	
	\newpage
	\section*{Análisis de contextos}
	
	\begin{multicols}{2}
		\raggedcolumns
		\noindent\textbf{Palabras más frecuentes cerca de \texttt{.</w>} (4209 apariciones):}
		\begin{verbatim}
			.</w>: 2320
			no</w>: 651
			la</w>: 445
			y</w>: 430
			pero</w>: 415
		\end{verbatim}
		
		\noindent\textbf{Palabras más frecuentes cerca de \texttt{,</w>} (3738 apariciones):}
		\begin{verbatim}
			y</w>: 2016
			pero</w>: 757
			la</w>: 720
			que</w>: 619
			el</w>: 585
		\end{verbatim}
		
		\noindent\textbf{Palabras más frecuentes cerca de \texttt{el</w>} (1077 apariciones):}
		\begin{verbatim}
			en</w>: 1224
			,</w>: 585
			que</w>: 477
			y</w>: 415
			.</w>: 408
		\end{verbatim}
		
		\noindent\textbf{Palabras más frecuentes cerca de \texttt{de</w>} (999 apariciones):}
		\begin{verbatim}
			la</w>: 1867
			los</w>: 670
			,</w>: 505
			las</w>: 483
			que</w>: 481
		\end{verbatim}
		
		\noindent\textbf{Palabras más frecuentes cerca de \texttt{a</w>} (857 apariciones):}
		\begin{verbatim}
			la</w>: 999
			,</w>: 458
			.</w>: 357
			y</w>: 263
			los</w>: 251
		\end{verbatim}
		
		\noindent\textbf{Palabras más frecuentes cerca de \texttt{la</w>} (738 apariciones):}
		\begin{verbatim}
			de</w>: 1867
			en</w>: 1115
			a</w>: 999
			,</w>: 720
			maga</w>: 463
		\end{verbatim}
		
		\noindent\textbf{Palabras más frecuentes cerca de \texttt{y</w>} (651 apariciones):}
		\begin{verbatim}
			,</w>: 2016
			.</w>: 430
			el</w>: 415
			la</w>: 393
			se</w>: 326
		\end{verbatim}
		
		\noindent\textbf{Palabras más frecuentes cerca de \texttt{que</w>} (642 apariciones):}
		\begin{verbatim}
			lo</w>: 790
			,</w>: 619
			se</w>: 573
			no</w>: 504
			de</w>: 481
		\end{verbatim}
		
		\noindent\textbf{Palabras más frecuentes cerca de \texttt{en</w>} (587 apariciones):}
		\begin{verbatim}
			el</w>: 1224
			la</w>: 1115
			,</w>: 459
			que</w>: 405
			.</w>: 374
		\end{verbatim}
	\end{multicols}
	
	
	\newpage
	\section*{Palabras que no se predijeron bien}
	\begin{multicols}{3}
		\footnotesize
		\begin{verbatim}
			.</w>              4209
			,</w>              3738
			el</w>             1077
			de</w>              999
			a</w>               857
			la</w>              738
			y</w>               651
			que</w>             642
			en</w>              587
			un</w>              388
			no</w>              385
			se</w>              354
			oliveira</w>        266
			del</w>             252
			lo</w>              196
			dijo</w>            123
			por</w>              73
			como</w>             66
			una</w>              48
			los</w>              42
			me</w>               31
			?</w>                28
			con</w>              24
			le</w>               16
			te</w>               14
			era</w>              13
			es</w>               12
			las</w>               9
			traveler</w>          9
			yo</w>                9
			horacio</w>           8
			así</w>               6
			todo</w>              6
			tu</w>                6
			hasta</w>             5
			eso</w>               5
			vos</w>               4
			ella</w>              4
			sí</w>                4
			ronald</w>            4
			al</w>                4
			o</w>                 4
			gekrepten</w>         4
			os</w>                3
			después</w>           3
			pero</w>              3
			más</w>               3
			algo</w>              3
			está</w>              3
			había</w>             3
			les</w>               3
			gregorovius</w>       3
			maga</w>              2
			babs</w>              2
			ese</w>               2
			gran</w>              2
			camas</w>             2
			cuando</w>            2
			noche</w>             2
			amor</w>              2
			qué</w>               2
			cosas</w>             2
			tiempo</w>            2
			decía</w>             2
			lucía</w>             1
			yes</w>               1
			re                    1
			porque</w>            1
			oh</w>                1
			ya</w>                1
			entonces</w>          1
			estás</w>             1
			ver                   1
			sin</w>               1
			esta</w>              1
			siempre</w>           1
			club</w>              1
			sé</w>                1
			just                  1
			muy</w>               1
			bajo</w>              1
			enormemente</w>       1
			ossip</w>             1
			nos</w>               1
			aba</w>               1
			célestin</w>          1
			podían</w>            1
			tanto</w>             1
			señora</w>            1
			jo                    1
			tarde</w>             1
			cualquier</w>         1
			frío</w>              1
			mucho</w>             1
			siquiera</w>          1
			tablón</w>            1
			su</w>                1
			dice</w>              1
			esperá</w>            1
			ser</w>               1
			claro</w>             1
			cuenta</w>            1
			kibbutz</w>           1
			sopa</w>              1
			bueno</w>             1
			otro</w>              1
			tan</w>               1
			aunque</w>            1
			e</w>                 1
			ahí</w>               1
			ventana</w>           1
			dos</w>               1
			eleg                  1
			ke</w>                1
			apenas</w>            1
			hace</w>              1
			malo</w>              1
			pieza</w>             1
			puerta</w>            1
			cama<</w>             1
			todavía</w>           1
			i                     1
			rojo</w>              1
			amarillo</w>          1
			horrible</w>          1
			efectos/w>            1
			estaba</w>            1
			hacía</w>             1
			ver</w>               1
			ojos</w>              1
			papeles</w>           1
			entrada</w>           1
			s                     1
			para</w>              1
			vida</w>              1
			café</w>              1
			
		\end{verbatim}
	\end{multicols}
	
\newpage	
	
\noindent A continuación, se presentan los mejores valores obtenidos para cada método evaluado, considerando las métricas \textbf{Top1} y \textbf{Top5}. Estos resultados permiten comparar el rendimiento general de los distintos métodos en términos de precisión de predicción de la palabra central.

\begin{figure}[htb]
	\centering
	\includegraphics[width=0.8\textwidth]{mejores_cbow}
	\caption{Comparación de las métricas \textbf{Top1} y \textbf{Top5} para los distintos métodos evaluados.}
	\label{fig:mejores_valores_modelos}
\end{figure}

\noindent Como se observa en la Figura~\ref{fig:mejores_valores_modelos}, los métodos fragmentados presentan comportamientos similares en \textbf{Top5}, pero sin dudas hay un claro ganados. El análisis de estas variaciones permite seleccionar los parámetros más adecuados y determinar qué configuraciones de entrenamiento logran un mejor equilibrio entre exactitud y generalización. Como prueba quedaría ver si continuando las agrupaciones de tokens con umbrales más chicos mejora el método.	
	
	
!! Muchas gracias ¡¡
	
\end{document}
